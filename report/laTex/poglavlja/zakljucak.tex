\section{Закључак}
У овом раду изучени су алгоритми који се користе за решавање судоку загонетке произвољне тежине (пропагација ограничења, претрага простора могућих решења), као и одређени приступи њиховој оптимизацији (бинарна представа могућих бројева ћелије, проналажење оптималног кандидата за даљу претрагу). Фокус је био на паралелизацији алгоритма за претрагу простора могућих решења. Проучаване имплементације користиле су се \textit{OpenMP} примитивима и \textit{task}-овима, као и \textit{OpenMPI}-ем и увеле неке нове приступе у односу на постојећа решења. Након мерења перформанси сваке од имплементацијa, дошло се до закључка да је секвенцијална имплементација ипак најбржа, због оптимизованости алгоритма за пропагирање ограничења и највећим делом \textit{overhead}-а који уводи потреба за координацијом нити и процеса.